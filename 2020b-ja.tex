% このファイルは日本語用です。
% 次の行は変更しないでください。
\documentclass[ja]{2020b}
%%%%%%%%%%%%%%%%%%%%%%%%%%%%%%%%%%%%%%%%%%%%%%%%%%%%%%%%%%%%%%%%
% 講演者についての情報
\PresenterInfo
%%%%%%%%%%%%%%%%%%%%%%%%%%%%%%%%
% 講演数(半角数字)
{1}
%%%%%%%%%%%%%%%%%%%%%%%%%%%%%%%%
% 氏名
{谷口暁星}
%%%%%%%%%%%%%%%%%%%%%%%%%%%%%%%%
% 氏(ひらがな, 氏名が英字の場合はalphabet)
{たにぐち}
%%%%%%%%%%%%%%%%%%%%%%%%%%%%%%%%
% 名(ひらがな, 氏名が英字の場合はalphabet)
{あきお}
%%%%%%%%%%%%%%%%%%%%%%%%%%%%%%%%
% 所属機関(機関名(◯◯大学、◯◯研究所、など)のみ)
{名古屋大学}
%%%%%%%%%%%%%%%%%%%%%%%%%%%%%%%%
% 会員種別(半角英小文字)
%   a=正会員(一般)
%   b=正会員(学生)
%   c=準会員(一般)
%   d=準会員(学生)
%   e=非会員(一般)〔企画セッションのみ〕
%   f=非会員(学生)〔企画セッションのみ〕
{a}
%%%%%%%%%%%%%%%%%%%%%%%%%%%%%%%%
% 会員番号(半角数字4桁)
%   入会申請中の場合、受付番号(半角英大文字+半角数字5桁)
{5892}
%%%%%%%%%%%%%%%%%%%%%%%%%%%%%%%%
% メールアドレス(半角)
{taniguchi@a.phys.nagoya-u.ac.jp}
%%%%%%%%%%%%%%%%%%%%%%%%%%%%%%%%%%%%%%%%%%%%%%%%%%%%%%%%%%%%%%%%
% 講演についての情報
\PaperInfo
%%%%%%%%%%%%%%%%%%%%%%%%%%%%%%%%
% 記者発表(半角英小文字)
%   申請する場合のみ「y」を記入
{}
%%%%%%%%%%%%%%%%%%%%%%%%%%%%%%%%
% 講演分野(半角)
%   [通常セッション]
%     M=太陽
%     N=恒星・恒星進化
%     P1=星・惑星形成(星形成)
%     P2=星・惑星形成(原始惑星系円盤)
%     P3=星・惑星形成(惑星系)
%     Q=星間現象
%     R=銀河
%     S=活動銀河核
%     T=銀河団
%     U=宇宙論
%     V1=観測機器(電波)
%     V2=観測機器(光赤外・重力波・その他)
%     V3=観測機器(X線・γ線)
%     W=コンパクト天体
%     X=銀河形成
%     Y=天文教育・広報普及・その他
%   [企画セッション]
%     Z1=超巨大ブラックホール研究の新展開:初撮像から形成進化の全貌解明へ
{V1}
%%%%%%%%%%%%%%%%%%%%%%%%%%%%%%%%
% 講演形式(半角英小文字)
%   a=口頭講演
{a}
%%%%%%%%%%%%%%%%%%%%%%%%%%%%%%%%
% キーワード(5つまで)
%   分野Y以外は PASJ keyword list から選択
{methods: data analysis}
{methods: observational}
{methods: statistical}
{submillimeter: galaxies}
{}
%%%%%%%%%%%%%%%%%%%%%%%%%%%%%%%%
% 題名
{スパースモデリングを使ったサブミリ波分光観測の高感度化}
%%%%%%%%%%%%%%%%%%%%%%%%%%%%%%%%
% 氏名及び所属(複数の場合は「, 」で区切)
{
    谷口暁星,
    田村陽一,
    萩本将都,
    戸上陽平 (名古屋大学),
    池田思朗 (統計数理研究所),
    竹腰達哉,
    吉村勇紀 (東京大学),
    川邊良平 (国立天文台)
}
%%%%%%%%%%%%%%%%%%%%%%%%%%%%%%%%%%%%%%%%%%%%%%%%%%%%%%%%%%%%%%%%
\begin{document}
%%%%%%%%%%%%%%%%%%%%%%%%%%%%%%%%%%%%%%%%%%%%%%%%%%%%%%%%%%%%%%%%
% 本文開始
%%%%%%%%%%%%%%%%%%%%%%%%%%%%%%%%%%%%%%%%%%%%%%%%%%%%%%%%%%%%%%%%

我々は、天体信号が観測データに占める割合が小さいという性質(疎性=スパース性)に着目することで、サブミリ波単一鏡によるポジションスイッチ観測を高感度化する解析手法を開発した。
観測の感度を制限する最大の要因は、天体信号の$10^{4}-10^{6}$倍ものパワーを持ち、最大数~Hzで時間変動する地球大気放射の雑音付加である。
ポジションスイッチ観測では天体と大気の空間2点を交互に観測することで、大気放射の時間変動をキャンセルしている。
ところが、スペクトル同士の減算によって感度が$\sqrt{2}$倍悪化するだけでなく、スイッチングより高速な時間変動をキャンセルできないため、達成可能な感度は観測装置が本来持つそれを下回る。

提案する解析手法は、スイッチングより高頻度($10^{0}-10^{1}$~Hz)に取得した時系列分光データ(2次元行列)を使用することで、観測手法に手を加えることなく大気の時間変動をモデルするものである。
この際、大気の時間変動は行列上で低ランクな成分として近似できる。
一方、天体信号は、観測帯域に占める割合が極端に大きくなければ行列上ではスパースな成分となる。
この2つの成分はGoDecアルゴリズム(Thou \& Tao 2011)で行列分解することが可能であり、感度を悪化させることなく大気放射の時間変動モデルと除去を達成する。

提案手法の最大の利点は、過去の観測も含め時系列分光データが取得可能な分光観測であれば、即座に感度を改善できる可能性を秘めていることである。
本講演では、メキシコのLMT~50~m鏡に搭載した2~mm帯受信機(川邊他2020年春季年会)で観測した、$z=2.55$のサブミリ波銀河のCO輝線分光データに提案手法を適用し、実際に$\sqrt{2}$倍以上の感度の向上が得られた実証例を紹介する。

%%%%%%%%%%%%%%%%%%%%%%%%%%%%%%%%%%%%%%%%%%%%%%%%%%%%%%%%%%%%%%%%
% 本文終了
%%%%%%%%%%%%%%%%%%%%%%%%%%%%%%%%%%%%%%%%%%%%%%%%%%%%%%%%%%%%%%%%
\end{document}
